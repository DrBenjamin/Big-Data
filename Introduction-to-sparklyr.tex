% Options for packages loaded elsewhere
\PassOptionsToPackage{unicode}{hyperref}
\PassOptionsToPackage{hyphens}{url}
%
\documentclass[
]{article}
\usepackage{amsmath,amssymb}
\usepackage{iftex}
\ifPDFTeX
  \usepackage[T1]{fontenc}
  \usepackage[utf8]{inputenc}
  \usepackage{textcomp} % provide euro and other symbols
\else % if luatex or xetex
  \usepackage{unicode-math} % this also loads fontspec
  \defaultfontfeatures{Scale=MatchLowercase}
  \defaultfontfeatures[\rmfamily]{Ligatures=TeX,Scale=1}
\fi
\usepackage{lmodern}
\ifPDFTeX\else
  % xetex/luatex font selection
\fi
% Use upquote if available, for straight quotes in verbatim environments
\IfFileExists{upquote.sty}{\usepackage{upquote}}{}
\IfFileExists{microtype.sty}{% use microtype if available
  \usepackage[]{microtype}
  \UseMicrotypeSet[protrusion]{basicmath} % disable protrusion for tt fonts
}{}
\makeatletter
\@ifundefined{KOMAClassName}{% if non-KOMA class
  \IfFileExists{parskip.sty}{%
    \usepackage{parskip}
  }{% else
    \setlength{\parindent}{0pt}
    \setlength{\parskip}{6pt plus 2pt minus 1pt}}
}{% if KOMA class
  \KOMAoptions{parskip=half}}
\makeatother
\usepackage{xcolor}
\usepackage[margin=1in]{geometry}
\usepackage{color}
\usepackage{fancyvrb}
\newcommand{\VerbBar}{|}
\newcommand{\VERB}{\Verb[commandchars=\\\{\}]}
\DefineVerbatimEnvironment{Highlighting}{Verbatim}{commandchars=\\\{\}}
% Add ',fontsize=\small' for more characters per line
\usepackage{framed}
\definecolor{shadecolor}{RGB}{248,248,248}
\newenvironment{Shaded}{\begin{snugshade}}{\end{snugshade}}
\newcommand{\AlertTok}[1]{\textcolor[rgb]{0.94,0.16,0.16}{#1}}
\newcommand{\AnnotationTok}[1]{\textcolor[rgb]{0.56,0.35,0.01}{\textbf{\textit{#1}}}}
\newcommand{\AttributeTok}[1]{\textcolor[rgb]{0.13,0.29,0.53}{#1}}
\newcommand{\BaseNTok}[1]{\textcolor[rgb]{0.00,0.00,0.81}{#1}}
\newcommand{\BuiltInTok}[1]{#1}
\newcommand{\CharTok}[1]{\textcolor[rgb]{0.31,0.60,0.02}{#1}}
\newcommand{\CommentTok}[1]{\textcolor[rgb]{0.56,0.35,0.01}{\textit{#1}}}
\newcommand{\CommentVarTok}[1]{\textcolor[rgb]{0.56,0.35,0.01}{\textbf{\textit{#1}}}}
\newcommand{\ConstantTok}[1]{\textcolor[rgb]{0.56,0.35,0.01}{#1}}
\newcommand{\ControlFlowTok}[1]{\textcolor[rgb]{0.13,0.29,0.53}{\textbf{#1}}}
\newcommand{\DataTypeTok}[1]{\textcolor[rgb]{0.13,0.29,0.53}{#1}}
\newcommand{\DecValTok}[1]{\textcolor[rgb]{0.00,0.00,0.81}{#1}}
\newcommand{\DocumentationTok}[1]{\textcolor[rgb]{0.56,0.35,0.01}{\textbf{\textit{#1}}}}
\newcommand{\ErrorTok}[1]{\textcolor[rgb]{0.64,0.00,0.00}{\textbf{#1}}}
\newcommand{\ExtensionTok}[1]{#1}
\newcommand{\FloatTok}[1]{\textcolor[rgb]{0.00,0.00,0.81}{#1}}
\newcommand{\FunctionTok}[1]{\textcolor[rgb]{0.13,0.29,0.53}{\textbf{#1}}}
\newcommand{\ImportTok}[1]{#1}
\newcommand{\InformationTok}[1]{\textcolor[rgb]{0.56,0.35,0.01}{\textbf{\textit{#1}}}}
\newcommand{\KeywordTok}[1]{\textcolor[rgb]{0.13,0.29,0.53}{\textbf{#1}}}
\newcommand{\NormalTok}[1]{#1}
\newcommand{\OperatorTok}[1]{\textcolor[rgb]{0.81,0.36,0.00}{\textbf{#1}}}
\newcommand{\OtherTok}[1]{\textcolor[rgb]{0.56,0.35,0.01}{#1}}
\newcommand{\PreprocessorTok}[1]{\textcolor[rgb]{0.56,0.35,0.01}{\textit{#1}}}
\newcommand{\RegionMarkerTok}[1]{#1}
\newcommand{\SpecialCharTok}[1]{\textcolor[rgb]{0.81,0.36,0.00}{\textbf{#1}}}
\newcommand{\SpecialStringTok}[1]{\textcolor[rgb]{0.31,0.60,0.02}{#1}}
\newcommand{\StringTok}[1]{\textcolor[rgb]{0.31,0.60,0.02}{#1}}
\newcommand{\VariableTok}[1]{\textcolor[rgb]{0.00,0.00,0.00}{#1}}
\newcommand{\VerbatimStringTok}[1]{\textcolor[rgb]{0.31,0.60,0.02}{#1}}
\newcommand{\WarningTok}[1]{\textcolor[rgb]{0.56,0.35,0.01}{\textbf{\textit{#1}}}}
\usepackage{longtable,booktabs,array}
\usepackage{calc} % for calculating minipage widths
% Correct order of tables after \paragraph or \subparagraph
\usepackage{etoolbox}
\makeatletter
\patchcmd\longtable{\par}{\if@noskipsec\mbox{}\fi\par}{}{}
\makeatother
% Allow footnotes in longtable head/foot
\IfFileExists{footnotehyper.sty}{\usepackage{footnotehyper}}{\usepackage{footnote}}
\makesavenoteenv{longtable}
\usepackage{graphicx}
\makeatletter
\def\maxwidth{\ifdim\Gin@nat@width>\linewidth\linewidth\else\Gin@nat@width\fi}
\def\maxheight{\ifdim\Gin@nat@height>\textheight\textheight\else\Gin@nat@height\fi}
\makeatother
% Scale images if necessary, so that they will not overflow the page
% margins by default, and it is still possible to overwrite the defaults
% using explicit options in \includegraphics[width, height, ...]{}
\setkeys{Gin}{width=\maxwidth,height=\maxheight,keepaspectratio}
% Set default figure placement to htbp
\makeatletter
\def\fps@figure{htbp}
\makeatother
\setlength{\emergencystretch}{3em} % prevent overfull lines
\providecommand{\tightlist}{%
  \setlength{\itemsep}{0pt}\setlength{\parskip}{0pt}}
\setcounter{secnumdepth}{-\maxdimen} % remove section numbering
\ifLuaTeX
  \usepackage{selnolig}  % disable illegal ligatures
\fi
\usepackage{bookmark}
\IfFileExists{xurl.sty}{\usepackage{xurl}}{} % add URL line breaks if available
\urlstyle{same}
\hypersetup{
  pdftitle={Introduction to sparklyr},
  hidelinks,
  pdfcreator={LaTeX via pandoc}}

\title{Introduction to sparklyr}
\author{}
\date{\vspace{-2.5em}}

\begin{document}
\maketitle

We will largely follow chapters \textbf{2} and \textbf{3} of
\textbf{Mastering Spark with R},
\href{https://therinspark.com/index.html}{https://therinspark.com}.

First install the following packages if you do not already have them
already, and load them with the \texttt{library()} function:

\begin{Shaded}
\begin{Highlighting}[]
\FunctionTok{library}\NormalTok{(sparklyr)}
\FunctionTok{library}\NormalTok{(dplyr)}
\FunctionTok{library}\NormalTok{(ggplot2)}
\FunctionTok{library}\NormalTok{(knitr)}
\end{Highlighting}
\end{Shaded}

\subsection{Preliminiaries}\label{preliminiaries}

If you are working on EIDF, first make sure that the default working
directory in RStudio is your folder. In RStudio, select Tools
-\textgreater{} Global Options. Change the default working directory to
be /work/eidf071/eidf071/.

To confirm the change has taken effect, close and then reopen RStudio,
and type getpw() into the console. It should show your working directory
correctly as above.

\subsection{Connecting}\label{connecting}

\begin{Shaded}
\begin{Highlighting}[]
\NormalTok{sc }\OtherTok{=} \FunctionTok{spark\_connect}\NormalTok{(}\AttributeTok{master =} \StringTok{\textquotesingle{}local\textquotesingle{}}\NormalTok{)}
\end{Highlighting}
\end{Shaded}

\section{Task 3}\label{task-3}

Which dplyr functions can be used to have a first look at the data? Run
them.

\begin{Shaded}
\begin{Highlighting}[]
\CommentTok{\#install.packages("nycflights13", "Lahman")}
\FunctionTok{library}\NormalTok{(dplyr)}
\NormalTok{iris\_tbl }\OtherTok{\textless{}{-}} \FunctionTok{copy\_to}\NormalTok{(sc, iris, }\AttributeTok{OVERWRITE =} \ConstantTok{TRUE}\NormalTok{)}
\NormalTok{flights\_tbl }\OtherTok{\textless{}{-}} \FunctionTok{copy\_to}\NormalTok{(sc, nycflights13}\SpecialCharTok{::}\NormalTok{flights, }\StringTok{"flights"}\NormalTok{, }\AttributeTok{OVERWRITE =} \ConstantTok{TRUE}\NormalTok{)}
\NormalTok{batting\_tbl }\OtherTok{\textless{}{-}} \FunctionTok{copy\_to}\NormalTok{(sc, Lahman}\SpecialCharTok{::}\NormalTok{Batting, }\StringTok{"batting"}\NormalTok{, }\AttributeTok{OVERWRITE =} \ConstantTok{TRUE}\NormalTok{)}
\FunctionTok{src\_tbls}\NormalTok{(sc)}
\end{Highlighting}
\end{Shaded}

\begin{verbatim}
## [1] "batting" "flights" "iris"
\end{verbatim}

\begin{Shaded}
\begin{Highlighting}[]
\NormalTok{flights\_tbl }\SpecialCharTok{\%\textgreater{}\%} \FunctionTok{filter}\NormalTok{(dep\_delay }\SpecialCharTok{==}\DecValTok{2}\NormalTok{)}
\end{Highlighting}
\end{Shaded}

\begin{verbatim}
## # Source:   SQL [?? x 19]
## # Database: spark_connection
##     year month   day dep_time sched_dep_time dep_delay arr_time sched_arr_time
##    <int> <int> <int>    <int>          <int>     <dbl>    <int>          <int>
##  1  2013     1     1      517            515         2      830            819
##  2  2013     1     1      542            540         2      923            850
##  3  2013     1     1      702            700         2     1058           1014
##  4  2013     1     1      715            713         2      911            850
##  5  2013     1     1      752            750         2     1025           1029
##  6  2013     1     1      917            915         2     1206           1211
##  7  2013     1     1      932            930         2     1219           1225
##  8  2013     1     1     1028           1026         2     1350           1339
##  9  2013     1     1     1042           1040         2     1325           1326
## 10  2013     1     1     1231           1229         2     1523           1529
## # i more rows
## # i 11 more variables: arr_delay <dbl>, carrier <chr>, flight <int>,
## #   tailnum <chr>, origin <chr>, dest <chr>, air_time <dbl>, distance <dbl>,
## #   hour <dbl>, minute <dbl>, time_hour <dttm>
\end{verbatim}

\begin{Shaded}
\begin{Highlighting}[]
\NormalTok{delay }\OtherTok{\textless{}{-}}\NormalTok{ flights\_tbl }\SpecialCharTok{\%\textgreater{}\%} 
  \FunctionTok{group\_by}\NormalTok{(tailnum) }\SpecialCharTok{\%\textgreater{}\%}
  \FunctionTok{summarise}\NormalTok{(}\AttributeTok{count =} \FunctionTok{n}\NormalTok{(), }\AttributeTok{dist =} \FunctionTok{mean}\NormalTok{(distance, }\AttributeTok{na.rm =} \ConstantTok{TRUE}\NormalTok{), }\AttributeTok{delay =} \FunctionTok{mean}\NormalTok{(arr\_delay, }\AttributeTok{na.rm =} \ConstantTok{TRUE}\NormalTok{)) }\SpecialCharTok{\%\textgreater{}\%}
  \FunctionTok{filter}\NormalTok{(count }\SpecialCharTok{\textgreater{}} \DecValTok{20}\NormalTok{, dist }\SpecialCharTok{\textless{}} \DecValTok{2000}\NormalTok{, }\SpecialCharTok{!}\FunctionTok{is.na}\NormalTok{(delay)) }\SpecialCharTok{\%\textgreater{}\%}
  \FunctionTok{collect}\NormalTok{()}

\FunctionTok{library}\NormalTok{(ggplot2)}
\FunctionTok{ggplot}\NormalTok{(delay, }\FunctionTok{aes}\NormalTok{(dist, delay)) }\SpecialCharTok{+}
  \FunctionTok{geom\_point}\NormalTok{(}\FunctionTok{aes}\NormalTok{(}\AttributeTok{size =}\NormalTok{ count), }\AttributeTok{alpha =} \DecValTok{1}\SpecialCharTok{/}\DecValTok{2}\NormalTok{) }\SpecialCharTok{+}
  \FunctionTok{geom\_smooth}\NormalTok{() }\SpecialCharTok{+}
  \FunctionTok{scale\_size\_area}\NormalTok{(}\AttributeTok{max\_size =} \DecValTok{2}\NormalTok{)}
\end{Highlighting}
\end{Shaded}

\begin{verbatim}
## `geom_smooth()` using method = 'gam' and formula = 'y ~ s(x, bs = "cs")'
\end{verbatim}

\includegraphics{Introduction-to-sparklyr_files/figure-latex/unnamed-chunk-3-1.pdf}

\section{Task 4}\label{task-4}

Count the occurrence of all the different values for the order variable
that are present in the dataset. Use kable to display any tables.

\begin{Shaded}
\begin{Highlighting}[]
\CommentTok{\# your code here}
\end{Highlighting}
\end{Shaded}

\section{Task 5}\label{task-5}

Select the columns name, order and brainwt. Filter to only those rows
where the order variable is Primates. Finally, arrange the result by
brainwt. Try to use the pipe operator \%\textgreater\% to cut down on
how many lines of code you write.

\begin{Shaded}
\begin{Highlighting}[]
\CommentTok{\# your code here}
\end{Highlighting}
\end{Shaded}

\section{Task 6}\label{task-6}

In this task, we will create a table of mean brain weight to body weight
ratio, grouped by the variable order. Create a new column called
brain\_body\_wt\_ratio that is equal to brainwt/bodywt. Group by the
variable order, and then calculate the mean for each group.

\begin{Shaded}
\begin{Highlighting}[]
\CommentTok{\# your code here}
\end{Highlighting}
\end{Shaded}

\section{Task 7}\label{task-7}

The total number of missing values in each column can be viewed using
the following code

\begin{Shaded}
\begin{Highlighting}[]
\NormalTok{sleep }\SpecialCharTok{\%\textgreater{}\%}
  \FunctionTok{summarise\_all}\NormalTok{(}\SpecialCharTok{\textasciitilde{}}\FunctionTok{sum}\NormalTok{(}\FunctionTok{as.integer}\NormalTok{(}\FunctionTok{is.na}\NormalTok{(.)))) }\SpecialCharTok{\%\textgreater{}\%}
  \FunctionTok{kable}\NormalTok{()}
\end{Highlighting}
\end{Shaded}

\begin{longtable}[]{@{}rrr@{}}
\toprule\noalign{}
extra & group & ID \\
\midrule\noalign{}
\endhead
\bottomrule\noalign{}
\endlastfoot
0 & 0 & 0 \\
\end{longtable}

Impute missing values for brainwt. Do this by creating a new column
called brainwt\_imputed, where NA values are replaced with the mean
value for brainwt. Verify the result by displaying the head of a table
with the columns name, brainwt and brainwt\_imputed. You may find the
ifelse or case\_when functions useful (Google them).

\begin{Shaded}
\begin{Highlighting}[]
\CommentTok{\# your code here}
\end{Highlighting}
\end{Shaded}

\section{Task 8}\label{task-8}

Use collect() and ggplot to make a horizontal bar chart with order on
the vertical axis and average sleep\_total on the horizontal axis. You
may find geom\_col from ggplot useful.

\begin{Shaded}
\begin{Highlighting}[]
\CommentTok{\# your code here}
\end{Highlighting}
\end{Shaded}


\end{document}
